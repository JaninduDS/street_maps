\documentclass[11pt, a4paper]{article}
\usepackage[utf8]{inputenc}
\usepackage[T1]{fontenc}
\usepackage[margin=1in]{geometry}
\usepackage{booktabs}
\usepackage{hyperref}
\usepackage{titlesec}
\usepackage{enumitem}
\usepackage{xcolor}

% Color definitions
\definecolor{primary}{RGB}{0, 70, 140}

% Hyperlink setup
\hypersetup{
    colorlinks=true,
    linkcolor=primary,
    filecolor=magenta,      
    urlcolor=primary,
}

% Section formatting
\titleformat{\section}
{\normalfont\Large\bfseries\color{primary}}
{\thesection}{1em}{}

\titleformat{\subsection}
{\normalfont\large\bfseries\color{primary}}
{\thesubsection}{1em}{}

% Meta data
\title{\textbf{Project Progress Report: Lumina Lanka}\\
\Large Smart Street Light Management System for Maharagama Urban Council}
\author{\textbf{Prepared for:} [Supervisor Name] \\
\textbf{From:} Lead Developer}
\date{\today}

\begin{document}

\maketitle

\section*{Executive Summary}
Lumina Lanka is a high-fidelity, cross-platform Smart Street Light Management System designed to streamline maintenance reporting and tracking using GPS technology. This solution delivers a seamless native experience across iOS, Android, Web, and Linux platforms, eliminating the need for complex server hardware. Phase 1 (Infrastructure \& UI) is now complete, establishing a robust foundation for scalable deployment.

\section{Technical Infrastructure}
The technical architecture of Lumina Lanka has been strategically chosen to minimize overhead while maximizing performance and scalability.

\subsection*{Serverless Architecture}
We have adopted a \textbf{Serverless Architecture} utilizing \textbf{Google Firebase} as the backend infrastructure. This decision ensures high availability and eliminates the need for maintaining traditional server infrastructure.

\begin{center}
    \textbf{No physical server hardware is required.}
\end{center}

\subsection*{Technology Stack}
\begin{itemize}
    \item \textbf{Geospatial Rendering:} The system leverages the \textbf{Google Maps Platform} to provide accurate, high-performance geospatial data rendering, essential for locating and managing street light infrastructure.
    \item \textbf{Cross-Platform Framework:} The application is built with \textbf{Flutter}, enabling us to compile native code for iOS, Android, Web, and Linux from a single codebase. This ensures consistent high performance and a unified user experience across all devices.
\end{itemize}

\section{Development Timeline: Phase 1}
The following timeline outlines the key milestones achieved during the first month of development.

\begin{table}[h]
\centering
\renewcommand{\arraystretch}{1.5}
\begin{tabular}{@{}c l p{9cm}@{}}
\toprule
\textbf{Week} & \textbf{Focus Area} & \textbf{Key Achievements} \\ \midrule
1 & Environment Setup & Configured Arch Linux development environment. Installed Flutter SDK, Android Studio, and the Linux Build Toolchain to ensure a robust development workflow. \\ 
2 & Core UI Engineering & Implemented "Next-Gen" Glassmorphism UI (inspired by iOS 26 concepts). Developed custom shaders and high-blur backdrops to achieve a premium, modern aesthetic. \\ 
3 & Map Integration & Integrated Google Maps Web API with hardware acceleration enabled. Implemented custom "Night Mode" mapping styles to enhance visibility and reduce eye strain during night operations. \\ 
4 & Backend Init & Initialized Firebase Project (\texttt{lumina-lanka}). Configured Firestore Database structure and linked API keys to secure connectivity between client and backend. \\ \bottomrule
\end{tabular}
\end{table}

\section{Current Status}
As of this report, Phase 1 is complete. The system's status is as follows:
\begin{itemize}
    \item \textbf{Platform Availability:} The application is successfully building and running on \textbf{Localhost (Web)} and \textbf{Linux Desktop}.
    \item \textbf{Map System:} Real-time map rendering is active and responsive.
    \item \textbf{Connectivity:} Secure connection to the Google Firestore database has been established and verified.
\end{itemize}

\end{document}
