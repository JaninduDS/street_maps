\documentclass[11pt, a4paper]{article}
\usepackage[utf8]{inputenc}
\usepackage[T1]{fontenc}
\usepackage[margin=1in]{geometry}
\usepackage{booktabs}
\usepackage{hyperref}
\usepackage{titlesec}
\usepackage{enumitem}
\usepackage{xcolor}
\usepackage{tabularx}

% Color definitions
\definecolor{primary}{RGB}{0, 70, 140}
\definecolor{secondary}{RGB}{80, 80, 80}

% Hyperlink setup
\hypersetup{
    colorlinks=true,
    linkcolor=primary,
    filecolor=magenta,      
    urlcolor=primary,
}

% Section formatting
\titleformat{\section}
{\normalfont\Large\bfseries\color{primary}}
{\thesection}{1em}{}

\titleformat{\subsection}
{\normalfont\large\bfseries\color{primary}}
{\thesubsection}{1em}{}

% Meta data
\title{\textbf{Project Progress Report: Lumina Lanka}\\
\Large Smart Street Light Management System for Maharagama Urban Council}
\author{\textbf{Prepared for:} Rohan Kariyawasam \\
\textbf{From:} Janindu de Silva}
\date{\today}

\begin{document}

\maketitle

\section*{Executive Summary}
Lumina Lanka is a high-fidelity, cross-platform Smart Street Light Management System designed to streamline maintenance reporting and tracking using GPS technology. This solution delivers a seamless native experience across iOS, Android, Web, and Linux platforms. Phase 1 is now complete, establishing the core infrastructure for scalable deployment.

\section{Technical Infrastructure}
The technical architecture maximizes performance and scalability while minimizing maintenance overhead.

\subsection*{Cloud Infrastructure}
The system operates on a serverless architecture powered by Google Firebase. This infrastructure manages data storage, authentication, and real-time synchronization through a fully managed cloud environment. This approach ensures high availability and automatic scaling by leveraging Google's global network resources.

\subsection*{Technology Stack}
\begin{itemize}
    \item \textbf{Geospatial Rendering:} The Google Maps Platform is utilized for accurate, high-performance geospatial data rendering, which is essential for locating and managing street light infrastructure.
    \item \textbf{Cross-Platform Framework:} The application is built with Flutter, enabling the compilation of native code for iOS, Android, Web, and Linux from a single codebase.
\end{itemize}

\section{Completed Progress (Phase 1)}
The following timeline outlines the key milestones achieved during the first month of development.

\begin{table}[h]
\centering
\renewcommand{\arraystretch}{1.5}
\begin{tabularx}{\textwidth}{@{}c l X@{}}
\toprule
\textbf{Week} & \textbf{Focus Area} & \textbf{Key Achievements} \\ \midrule
1 & Environment Setup & Configured Arch Linux development environment, installed Flutter SDK, Android Studio, and verified Linux Toolchain. \\ 
2 & Core UI Engineering & Implemented Glassmorphism UI concepts. Developed custom shaders and high-blur backdrops for the application interface. \\ 
3 & Map Integration & Integrated Google Maps Web API with hardware acceleration. Implemented custom night mode mapping styles for improved visibility. \\ 
4 & Backend Init & Initialized the Firebase Project. Configured the Firestore Database structure and established secure API connectivity. \\ \bottomrule
\end{tabularx}
\end{table}

\section{Upcoming Implementation Roadmap}
The project will proceed through the following phases to achieve full deployment in the Maharagama region.

\subsection*{Phase 2: Data Digitization}
\textit{Objective: Populate the system with physical assets.}
\begin{itemize}
    \item \textbf{Precise Geotagging:} Implementation of high-accuracy GPS pinning logic.
    \item \textbf{Asset Classification:} Data entry forms for classifying pole and bulb types.
    \item \textbf{Volunteer Access:} Deployment of the mapping module for Ward-level data tagging.
\end{itemize}

\subsection*{Phase 3: Public Reporting and Administration}
\textit{Objective: Enable citizen engagement and council oversight.}
\begin{itemize}
    \item \textbf{Public Interface:} Launching the reporting interface for users to flag faulty street lights.
    \item \textbf{Council Dashboard:} A web-based analytical view for identifying high-failure zones.
    \item \textbf{Ward Filtering:} Filtering logic to view issues by specific administrative boundaries.
\end{itemize}

\subsection*{Phase 4: Maintenance Operations}
\textit{Objective: Closing the loop on repairs and warranty.}
\begin{itemize}
    \item \textbf{Job Cards:} Automated assignment of faulty lights to maintenance staff.
    \item \textbf{Warranty Tracking:} Data entry of serial numbers during repair to automate stock records.
    \item \textbf{Resolution:} System logic to update light status upon job completion.
\end{itemize}

\subsection*{Phase 5: Pilot Deployment}
\textit{Objective: Live testing in the Maharagama Urban Council area.}
\begin{itemize}
    \item Field testing with public security committee volunteers.
    \item Stress testing the backend with real-time reporting data.
    \item Final calibration of GPS coordinates for Ward boundaries.
\end{itemize}

\section{Current Status Summary}
As of this report, the foundational infrastructure is operational. The application runs successfully on Localhost (Web) and Linux Desktop, featuring active real-time map rendering and a secured database connection.

\end{document}
